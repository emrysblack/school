% Created 2016-06-02 Thu 21:43
\documentclass[11pt]{article}
\usepackage[utf8]{inputenc}
\usepackage[T1]{fontenc}
\usepackage{graphicx}
\usepackage{longtable}
\usepackage{hyperref}


\title{Chapter 6 Exercises}
\author{<Your Name Goes Here>}
\date{02 June 2016}

\begin{document}

\maketitle


\section*{Exercise 1 on page 216}
\label{sec-1}


Solve the following system by Gaussian Elimination:


\begin{center}
\begin{tabular}{llr}
 x$_1$ + x$_2$ + x$_3$   &  =  &  2  \\
 2x$_1$ + x$_2$ + x$_3$  &  =  &  3  \\
 x$_1$ - x$_2$ + 3x$_3$  &  =  &  8  \\
\end{tabular}
\end{center}



\emph{Up to 10 points are possible for a correct answer with a good analysis.}

\subsection*{Solution}
\label{sec-1.1}



\begin{center}
\begin{tabular}{rrrlr}
 1  &   1  &  1  &  =  &  2  \\
 2  &   1  &  1  &  =  &  3  \\
 1  &  -1  &  3  &  =  &  8  \\
\end{tabular}
\end{center}





\begin{center}
\begin{tabular}{rrrlr}
 1  &  1  &  1  &  =  &   2  \\
 1  &  0  &  0  &  =  &   1  \\
 2  &  0  &  4  &  =  &  10  \\
\end{tabular}
\end{center}





\begin{center}
\begin{tabular}{rrrlr}
 1  &  0  &  0  &  =  &  1  \\
 1  &  1  &  1  &  =  &  2  \\
 0  &  0  &  4  &  =  &  8  \\
\end{tabular}
\end{center}





\begin{center}
\begin{tabular}{rrrlr}
 1  &  0  &  0  &  =  &  1  \\
 0  &  1  &  1  &  =  &  1  \\
 0  &  0  &  1  &  =  &  2  \\
\end{tabular}
\end{center}





\begin{center}
\begin{tabular}{rrrlr}
 1  &  0  &  0  &  =  &   1  \\
 0  &  1  &  0  &  =  &  -1  \\
 0  &  0  &  1  &  =  &   2  \\
\end{tabular}
\end{center}





\begin{center}
\begin{tabular}{lll}
 x$_1$ = 1  &  x$_2$ = -1  &  x$_3$ = 2  \\
\end{tabular}
\end{center}



\section*{Exercise 2 on page 216}
\label{sec-2}


\textbf{a.} Solve the system of the previous question by the \emph{LU} decomposition method.

\textbf{b.} From the standpoint of general algorithm design techniques, how would
     you classify the \emph{LU} decomposition method?

\emph{Up to 10 points are possible for a correct answer with all the steps shown for part *a*.}

\emph{Up to 2 points are possible for a correct answer, well explained, for part *b*.}

\subsection*{Solution}
\label{sec-2.1}

\subsubsection*{a}
\label{sec-2.1.1}

\subsubsection*{b}
\label{sec-2.1.2}

Transform and Conquer - We are not reducing
\section*{Exercise 3 on page 216}
\label{sec-3}


Solve the system of Problem 1 by computing the inverse of its
coefficient matrix and then multiplying it by the right-hand side
vector.

\emph{Up to 10 points are possible for a correct answer with all the steps shown.}

\begin{center}
\begin{tabular}{rrrlrrr}
 1  &   1  &  1  &  :  &  1  &  0  &  0  \\
 2  &   1  &  1  &  :  &  0  &  1  &  0  \\
 1  &  -1  &  3  &  :  &  0  &  0  &  1  \\
\end{tabular}
\end{center}





\begin{center}
\begin{tabular}{rrrlrrr}
 -1  &   0  &  0  &  :  &  1  &  -1  &  0  \\
  2  &   1  &  1  &  :  &  0  &   1  &  0  \\
  1  &  -1  &  3  &  :  &  0  &   0  &  1  \\
\end{tabular}
\end{center}





\begin{center}
\begin{tabular}{rrrlrrr}
 -1  &   0  &  0  &  :  &  1  &  -1  &  0  \\
  0  &   1  &  1  &  :  &  2  &  -1  &  0  \\
  1  &  -1  &  3  &  :  &  0  &   0  &  1  \\
\end{tabular}
\end{center}





\begin{center}
\begin{tabular}{rrrlrrr}
 -1  &   0  &  0  &  :  &  1  &  -1  &  0  \\
  0  &   1  &  1  &  :  &  2  &  -1  &  0  \\
  0  &  -1  &  3  &  :  &  1  &  -1  &  1  \\
\end{tabular}
\end{center}





\begin{center}
\begin{tabular}{rrrlrrr}
 1  &   0  &  0  &  :  &  -1  &   1  &  0  \\
 0  &   1  &  1  &  :  &   2  &  -1  &  0  \\
 0  &  -1  &  3  &  :  &   1  &  -1  &  1  \\
\end{tabular}
\end{center}





\begin{center}
\begin{tabular}{rrrlrrr}
 1  &  0  &  0  &  :  &  -1  &   1  &  0  \\
 0  &  1  &  1  &  :  &   2  &  -1  &  0  \\
 0  &  0  &  4  &  :  &   3  &  -2  &  1  \\
\end{tabular}
\end{center}





\begin{center}
\begin{tabular}{rrrlrrr}
 1  &  0  &  0  &  :  &                 -1  &                   1  &                  0  \\
 0  &  1  &  1  &  :  &                  2  &                  -1  &                  0  \\
 0  &  0  &  1  &  :  &  \( \frac{3}{4} \)  &  -\( \frac{2}{4} \)  &  \( \frac{1}{4} \)  \\
\end{tabular}
\end{center}





\begin{center}
\begin{tabular}{rrrllll}
 1  &  0  &  0  &  :  &  -1                 &  1                   &  0                   \\
 0  &  1  &  0  &  :  &  \( \frac{5}{4} \)  &  -\( \frac{2}{4} \)  &  -\( \frac{1}{4} \)  \\
 0  &  0  &  1  &  :  &  \( \frac{3}{4} \)  &  -\( \frac{2}{4} \)  &  \( \frac{1}{4} \)   \\
\end{tabular}
\end{center}


\section*{Bonus Exercises}
\label{sec-4}


\emph{Up to 30 bonus points are possible for good answers, well explained, for Exercises 4, 5 and 6 on page 206.}

\end{document}
