% Created 2016-05-25 Wed 22:24
\documentclass[11pt]{article}
\usepackage[utf8]{inputenc}
\usepackage[T1]{fontenc}
\usepackage{graphicx}
\usepackage{longtable}
\usepackage{hyperref}


\title{Chapter 5 Exercises}
\author{Brady Field}
\date{25 May 2016}

\begin{document}

\maketitle



\section*{Partners in crime}
\label{sec-1}

Thom Allen - helped with max algorithm and classifying growths

\section*{Exercise 1 on page 174}
\label{sec-2}


\textbf{a.} Write a pseudocode for a divide-and-conquer algorithm for finding
a position of the largest element in an array of \emph{n} numbers.

\textbf{b.} What will be your algorithm's output for arrays with several
elements of the largest value?

\textbf{c.} Set up and solve a recurrence relation for the number of key
comparisons made by your algorithm.

\textbf{d.} How does this algorithm compare with the brute-force algorithm
for this problem?

\subsection*{Solution}
\label{sec-2.1}

\textbf{a.}

\begin{verbatim}
maxPos(list, l, r)
IF l == r
   RETURN l
SET mid to (l + r) / 2
SET left to maxPos(list,l,mid)
SET right to maxPos(list,mid+1,r)
IF list of left > list of right
   RETURN left
ELSE
   RETURN right

\end{verbatim}


\textbf{b.} The largest array index of all possible max values

\textbf{c.} assuming not ints\ldots{}

\(T(n) = 2T(n/2) + 2^0, T(1) = 0\)

\(2T(n/2) = 2^2T(n/2^2) + 2^1\)

\(2^2T(n/2^2) = 2^3T(n/2^3) + 2^2\)

\ldots{}

\(2^{k-1}T(n/2^{k-1}) = 2^kT(n/2^k) + 2^{k-1}\)

\(T(n) = 2^kT(1) + $$\sum_{n=0}^{k-1} 2^i $$ \)

\(T(n) = 2^k(0) + 2^k - 1, 2^k = n \)

\(T(n) = n - 1\)

\textbf{d.} They are the same. Both need to check every position exactly once

\section*{Exercise 5 on page 175}
\label{sec-3}


Find the order of growth for solutions of the following recurrences.

\textbf{a.} \(T(n) = 4T(n/2) + n, T(1) = 1\)

\textbf{b.} \(T(n) = 4T(n/2) + n^2, T(1) = 1\)

\textbf{c.} \(T(n) = 4T(n/2) + n^3, T(1) = 1\)

\textbf{NOTE:} No exercises are assigned for the remaining sections 5.2-5.5.

\subsection*{Solution}
\label{sec-3.1}



\(\Theta(n^d)\) if a \(< b^d\)

\(\Theta(n^dlogn)\) if a \(= b^d\)

\(\Theta(n^{\log_b a})\) if a \(> b^d\)

\textbf{a.}
a = 4, b = 2, d = 1, \(4 > 2^1\), \(\Theta(n^{\log_b a})\)

\textbf{b.}
a = 4, b = 2, d = 2, \(4 = 2^2\), \(\Theta(n^dlogn)\)

\textbf{c.}
a = 4, b = 2, d = 3, \(4 < 2^3\), \(\Theta(n^d)\)

\end{document}
