% Created 2016-06-09 Thu 22:46
\documentclass[11pt]{article}
\usepackage[utf8]{inputenc}
\usepackage[T1]{fontenc}
\usepackage{graphicx}
\usepackage{longtable}
\usepackage{hyperref}


\title{Chapter 7 Exercises}
\author{Brady Field}
\date{09 June 2016}

\begin{document}

\maketitle


\section*{Partners in crime}
\label{sec-1}

Landon - helped clarify what a few of the questions were asking, and
helped catch me up on what I missed after I had to leave class on
Friday.

Matthew - helped with linking image of 2-3-4 tree, and helped
with the top-down approach.

Thom - helped with the top-down insertion of the 2-3-4 tree.

\section*{Solution to Exercise 1 on page 274}
\label{sec-2}


\textbf{a.} open hash table


\begin{center}
\begin{tabular}{lllllllllll}
 [ 0 ]  &  [ 1 ]       &  [ 2 ]  &  [ 3 ]  &  [ 4 ]  &  [ 5 ]  &  [ 6 ]  &  [ 7 ]  &  [ 8 ]       &  [ 9 ]       &  [ 10 ]  \\
        &  \downarrow  &         &         &         &         &         &         &  \downarrow  &  \downarrow  &          \\
        &  56          &         &         &         &         &         &         &  30          &  20          &          \\
        &              &         &         &         &         &         &         &  \downarrow  &  \downarrow  &          \\
        &              &         &         &         &         &         &         &  19          &  75          &          \\
        &              &         &         &         &         &         &         &              &  \downarrow  &          \\
        &              &         &         &         &         &         &         &              &  31          &          \\
\end{tabular}
\end{center}



\textbf{b.} largest number of key comparisons

3 comparisons if looking for entry 31

\textbf{c.} average number of key comparisons

If we look for all the entries we get (1 + 1 + 1 + 2 + 3 + 2)/6 = 5/3
comparisons.

\section*{Solution to Exercise 2 on page 274}
\label{sec-3}


\textbf{a.} closed hash table


\begin{center}
\begin{tabular}{lllllllllll}
 [ 0 ]       &  [ 1 ]       &  [ 2 ]       &  [ 3 ]  &  [ 4 ]  &  [ 5 ]  &  [ 6 ]  &  [ 7 ]  &  [ 8 ]       &  [ 9 ]       &  [ 10 ]      \\
 \downarrow  &  \downarrow  &  \downarrow  &         &         &         &         &         &  \downarrow  &  \downarrow  &  \downarrow  \\
 31          &  56          &  19          &         &         &         &         &         &  30          &  20          &  75          \\
\end{tabular}
\end{center}



\textbf{b.} largest number of key comparisons

6 comparisons if looking for entry 19

\textbf{c.} average number of key comparisons

if we look for all the entries we get (1 + 1 + 1 + 2 + 3 + 6)/6 = 7/3
comparisons.

\section*{Solution to Exercise 3 on page 275}
\label{sec-4}


There would be way too many collisions to make the hash
effective. There are only 26 buckets for data, and far too many
possible words to fit nicely into only 26 buckets.

\section*{Solution to Exercise 4 on page 275}
\label{sec-5}


\left ( \frac{1}{m} \right )^{n} * {m} = \frac{1}{m^{n-1}}

1/m is the probability of hitting a specific
bucket. We raise to the n for the probability of hitting every
time. We then multiply it by m to extend it to any single bucket.

\section*{Solution to Exercise 3 on page 279}
\label{sec-6}


Following the technique outlined in the book, I used the formula 

\left
$\lfloor$ log\{_{\left $\lceil$ m/2 \right \rceil}\frac{n+1}{4}} \right
$\rfloor$ + 1

with 100 million for n, and got the height from plugging in values for
m in wolfram alpha. I tried going up by 100 each time until I
hit 600. I then tested 550, 580, 590, 585, 583, and 584 for m. This
allowed me to discover that 585 was the minimum order for the desired
height of 3.

\section*{Solution to Exercise 6 on page 279}
\label{sec-7}


\textbf{a.} \href{https://drive.google.com/file/d/10gBOSN8BeleyW8wsbETAueg0bg20tyrKkg/view}{Tree}. We did a step-by-step construction of the tree using the
top-down method.

\textbf{b.} Using the top-down approach, we have a maximum of one split per
insertion. This keeps things simple. The downside is that our tree may
not have an optimal height compared to another method.

\end{document}
